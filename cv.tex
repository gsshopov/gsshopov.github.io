\documentclass[a4paper, 12pt, sans, colorlinks, allcolors=color1]{moderncv}
\moderncvcolor{blue}
\moderncvstyle{classic}

\usepackage{fontspec}
\setsansfont{CMU Sans Serif}

\usepackage[bulgarian, british]{babel}

\usepackage[autostyle]{csquotes}
\DeclareQuoteStyle{bulgarian}
  {\quotedblbase}
  {\textquotedblleft}
  [0.05em]
  {\quotesinglbase}
  {\textquoteleft}
\DeclareQuoteAlias{english}{british}

\DeclareRobustCommand{\trans}[2]{%
  \iflanguage{bulgarian}%
    {#1}%
    {#2}%
}

\usepackage[margin=1in]{geometry}
\usepackage{microtype}

\usepackage[maxnames=100, sorting=none]{biblatex}

% Make the title of article, inproceedings and misc entries a link
\newbibmacro{string+urldoi}[1]{%
  \iffieldundef{url}{%
    \iffieldundef{doi}{%
        #1%
    }{%
      \href{http://dx.doi.org/\thefield{doi}}{#1}%
    }%
  }{%
    \href{\thefield{url}}{#1}%
  }%
}
\DeclareFieldFormat[article, inproceedings, misc]{title}{%
  \usebibmacro{string+urldoi}{#1}%
}

% Change arXiv formatting
\DeclareFieldFormat{eprint}{%
  \printfield{eprinttype}: #1 [\printfield{eprintclass}]%
}


% Simple driver for inproceedings
\DeclareBibliographyDriver{inproceedings}{%
  \printnames{author}. \\
  \printfield{title}. \\
  In \printfield{booktitle}, \printfield{year}.%
}

% Simple driver for article
\DeclareBibliographyDriver{article}{%
  \printnames{author}. \\
  \printfield{title}. \\
  In \printfield{journaltitle}, \printfield{year}.%
}

% Simple driver for misc (e.g., arXiv preprints)
\DeclareBibliographyDriver{misc}{%
  \printnames{author}. \\
  \printfield{title}. \\
  In \printfield{eprint}, \printfield{year}.
}

% Make left margin and label separator consistent with moderncv
\defbibenvironment{bibliography}{%
  \list{%
    \printtext[labelnumberwidth]{%
      \printfield{prefixnumber}%
      \printfield{labelnumber}%
    }%
  }{%
    \setlength{\labelwidth}{\labelnumberwidth}%
    \setlength{\leftmargin}{\hintscolumnwidth}%
    \setlength{\labelsep}{\separatorcolumnwidth}%
    \addtolength{\leftmargin}{\labelsep}%
    \setlength{\itemsep}{\bibitemsep}%
    \setlength{\parsep}{\bibparsep}%
  }%
  \renewcommand*{\makelabel}[1]{\hss##1}%
}{\endlist}{\item}


\addbibresource{publications.bib}

\title{%
  \trans{%
    Автобиография%
  }{%
    Curriculum Vitae%
  }%
}

\name{%
  \trans{%
    Георги%
  }{%
    Georgi%
  }%
}{%
  \trans{%
    Шопов%
  }{%
    Shopov%
  }%
}

\address{%
  \trans{%
    ул. ,,Акад. Георги Бончев``, блок 2%
  }{%
    Acad. Georgi Bonchev St., Block 2%
  }%
}{%
  \trans{%
    1113 София, България%
  }{%
    1113 Sofia, Bulgaria%
  }%
}{}

\email{gshopov@lml.bas.bg}

\begin{document}
  \makecvtitle
  \vspace*{-1cm}

  \section{%
    \trans{%
      Научни интереси%
    }{%
      Research Interests%
    }%
  }

  \cvlistitem{%
    \trans{%
      Формални езици и теория на автоматите%
    }{%
      Formal Languages and Automata Theory%
    }%
  }

  \cvlistitem{%
    \trans{%
      Структури от данни и алгоритми%
    }{%
      Data Structures and Algorithms%
    }%
  }

  \cvlistitem{%
    \trans{%
      Машинно обучение%
    }{%
      Machine Learning%
    }%
  }

  \section{%
    \trans{%
      Образование%
    }{%
      Education%
    }%
  }

  \cventry{%
    \rlap{2020--}\hphantom{2017--2019}%
  }{%
    \trans{%
      Доктор по компютърни науки%
    }{%
      Ph.D. in Computer Science%
    }%
  }{%
    \trans{%
      ИИКТ, Българска академия на науките%
    }{%
      IICT, Bulgarian Academy of Sciences%
    }%
  }{}{}{%
    \trans{%
      Дисертация%
    }{%
      Thesis%
    }:
    \href{%
      https://github.com/gsshopov/gsshopov.github.io/releases/download/v1.0/%
      phd-thesis.pdf%
    }{%
      \trans{%
        Крайни преобразуватели за представяне на езикови модели%
      }{%
        Finite Transducers for Representation of Language Models%
      }%
    } \\
    \trans{%
      Ръководител: проф. д.н. Стоян Михов%
    }{%
      Supervisor: Prof. D.Sc. Stoyan Mihov%
    }%
  }

  \cventry{%
    2017--2019%
  }{%
    \trans{%
      Магистър по компютърна лингвистика%
    }{%
      M.Sc. in Computational Linguistics%
    }%
  }{%
    \trans{%
      ФМИ, Софийски университет%
    }{%
      FMI, Sofia University%
    }%
  }{}{}{%
    \trans{%
      Дипломна работа%
    }{%
      Thesis%
    }:
    \href{%
      https://github.com/gsshopov/gsshopov.github.io/releases/download/v1.0/%
      msc-thesis.pdf%
    }{%
      \trans{%
        Построяване на \(f\)-преобразувател на ниво символи за представяне на
        езикови модели%
      }{%
        Construction of Character-Level \(f\)-Transducers for Representation of
        Language Models%
      }%
    } \\
    \trans{%
      Ръководител: проф. д.н. Стоян Михов%
    }{%
      Supervisor: Prof. D.Sc. Stoyan Mihov%
    }%
  }

  \cventry{%
    2012--2017%
  }{%
    \trans{%
      Бакалавър по компютърни науки%
    }{%
      B.Sc. in Computer Science%
    }%
  }{%
    \trans{%
      ФМИ, Софийски университет%
    }{%
      FMI, Sofia University%
    }%
  }{}{}{}

  \section{%
    \trans{%
      Професионален опит%
    }{%
      Professional Experience%
    }%
  }

  \cventry{%
    2020--2025%
  }{%
    \trans{%
      Асистент%
    }{%
      Assistant Professor%
    }%
  }{%
    \trans{%
      ИИКТ, Българска академия на науките%
    }{%
      IICT, Bulgarian Academy of Sciences%
    }%
  }{}{}{}

  \cventry{%
    2018--2019%
  }{%
    \trans{%
      Програмист%
    }{%
      Programmer%
    }%
  }{%
    \trans{%
      ИИКТ, Българска академия на науките%
    }{%
      IICT, Bulgarian Academy of Sciences%
    }%
  }{}{}{}

  \section{%
    \trans{%
      Преподавателска дейност%
    }{%
      Teaching Experience%
    }%
  }

  \cventry{}{%
    \trans{%
      Семинарни упражнения%
    }{%
      Teaching Assistant%
    }%
  }{%
    \trans{%
      ФМИ, Софийски университет%
    }{%
      FMI, Sofia University%
    }%
  }{}{}{}

  \cvitem{%
    2020--2021%
  }{%
    \trans{%
      Приложения на крайните автомати%
    }{%
      Applications of Finite Automata%
    }%
  }

  \cvitem{%
    2017--2019%
  }{%
    \trans{%
      Езици, автомати и изчислимост%
    }{%
      Languages, Automata and Computability%
    }%
  }

  \cvitem{%
    2014%
  }{%
    \trans{%
      Обектно-ориентирано програмиране%
    }{%
      Object-Oriented Programming%
    }%
  }

  \cvitem{%
    2013%
  }{%
    \trans{%
      Увод в програмирането%
    }{%
      Introduction to Programming%
    }%
  }

  \section{%
    \trans{%
      Научноприложни проекти%
    }{%
      Research Projects%
    }%
  }

  \cvitem{%
    2023%
  }{%
    \emph{%
      \trans{%
        Изследване на концепция за автоматично озвучаване на документални
        филми%
      }{%
        Study of a Concept for Automatic Dubbing of Documentaries%
      }%
    },
    \trans{%
      Финансиран от Доли Медия Студио ЕООД%
    }{%
      Funded by Doli Media Studio Ltd.%
    }%
  }

  \cvitem{%
    2023%
  }{%
    \emph{%
      \trans{%
        Консултации за разработване на невронна мрежа за разпознаване на
        реклами във видео поток%
      }{%
        Consultancy for the Development of a Neural Network for Recognising
        Advertisements in Video Streams%
      }%
    },
    \trans{%
      Финансиран от Ейч-Тек ЕООД%
    }{%
      Funded by H-Tech Ltd.%
    }%
  }

  \cvitem{%
    2021--2023%
  }{%
    \emph{%
      \trans{%
        Научна консултация с предмет ,,Разработване и поддържане на платформа
        като онлайн уеб-базирана система за езикови ресурси ориентирани към
        използването на българския книжовен език като официален`` (БЕРОН)%
      }{%
        Scientific Consultation on the Subject ``Development and Maintenance of
        a Platform as an Online Web-Based System for Language Resources
        Designed for the Use of Literary Bulgarian as the Official Language''
        (BERON)%
      }%
    },
    \trans{%
      Финансиран от Министерството на образованието и науката на България%
    }{%
      Funded by the Ministry of Education and Science of Bulgaria%
    }%
  }

  \cvitem{%
    2021--2022%
  }{%
    \emph{%
      \trans{%
        Разработване на синтезатор на българска реч за нуждите на хората със
        зрителни увреждания%
      }{%
        Development of a Bulgarian Speech Synthesiser for the Needs of People
        with Visual Impairments%
      }%
    },
    \trans{%
      Финансиран от Съюза на слепите в България%
    }{%
      Funded by the Union of the Blind in Bulgaria%
    }%
  }

  \cvitem{%
    2018--2022%
  }{%
    \emph{%
      \trans{%
        Национална научна програма ,,Електронно здравеопазване в България``
        (ННП еЗдраве), Работен пакет 5: ,,Създаване на прототип за компютърно
        подпомагане на изготвянето на медицинска документация (диктофон за
        български език)``%
      }{%
        National Research Program ``Electronic Health in Bulgaria'' (NSP
        eHealth), Work Package 5: ``Development of a Prototype for
        Computer-Aided Preparation of Medical Documentation (Voice Recorder for
        the Bulgarian Language)''%
      }%
    },
    \trans{%
      Финансиран от Министерството на образованието и науката на България%
    }{%
      Funded by the Ministry of Education and Science of Bulgaria%
    }%
  }

  \section{
    \trans{%
      Публикации%
    }{%
      Publications%
    }%
  }

  \selectlanguage{british}
  \nocite{*}
  \printbibliography[heading=none]

\end{document}

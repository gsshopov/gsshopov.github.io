\documentclass[12pt,a4paper,sans,colorlinks]{moderncv}
\moderncvcolor{blue}
\moderncvstyle{classic}

\usepackage[margin=0.95in]{geometry}
\usepackage[utf8]{inputenc}
\usepackage[T2A]{fontenc}
\usepackage[english,bulgarian]{babel}

\usepackage{enumitem}
\newlist{cvenumerate}{enumerate}{1}
\setlist[cvenumerate]{label=[\arabic*],leftmargin=\hintscolumnwidth+\separatorcolumnwidth,labelsep=\separatorcolumnwidth}

\AtBeginDocument{\hypersetup{allcolors={color1}}}

\name{Георги}{Шопов}
\title{Автобиография}
\address{ул. ,,Акад. Георги Бончев``, блок 2}{1113 София, България}{}
\email{gshopov@lml.bas.bg}

\begin{document}
\fancyfoot[r]{\color{color2}\pagenumberfont\strut\thepage/\pageref*{lastpage}}

\makecvtitle
\vspace*{-1cm}

\section{Научни интереси}
\cvlistitem{Формални езици и теория на автоматите}
\cvlistitem{Структури от данни и алгоритми}
\cvlistitem{Машинно обучение}

\section{Образование}
\cventry{\rlap{2020--}\hphantom{2017--2019}}{Доктор по Компютърни науки}{ИИКТ, Българска академия на науките}{}{}{Дисертация: \href{https://github.com/gsshopov/gsshopov.github.io/releases/download/v1.0/phd-thesis.pdf}{Крайни преобразуватели за представяне на езикови модели} \newline Ръководител: проф. д.н. Стоян Михов}
\cventry{2017--2019}{Магистър по Компютърна лингвистика}{ФМИ, Софийски университет}{}{}{Дипломна работа: \href{https://github.com/gsshopov/gsshopov.github.io/releases/download/v1.0/msc-thesis.pdf}{Построяване на \(f\)-преобразувател на ниво символи за представяне на езикови модели} \newline Ръководител: проф. д.н. Стоян Михов}
\cventry{2012--2017}{Бакалавър по Компютърни науки}{ФМИ, Софийски университет}{}{}{}

\section{Професионален опит}
\cventry{2020--2025}{Асистент}{ИИКТ, Българска академия на науките}{}{}{}
\cventry{2018--2019}{Програмист}{ИИКТ, Българска академия на науките}{}{}{}

\section{Преподавателска дейност}
\cventry{}{Семинарни упражнения}{ФМИ, Софийски университет}{}{}{}
\cvitem{2020--2021}{Приложения на крайните автомати}
\cvitem{2017--2019}{Езици, автомати и изчислимост}
\cvitem{2014}{Обектно-ориентирано програмиране}
\cvitem{2013}{Увод в програмирането}

\section{Научноприложни проекти}
\cvitem{2023}{\emph{Изследване на концепция за автоматично озвучаване на документални филми}, Финансиран от Доли Медия Студио ЕООД}
\cvitem{2023}{\emph{Консултации за разработване на невронна мрежа за разпознаване на реклами във видео поток}, Финансиран от Ейч-Тек ЕООД}
\cvitem{2021--2023}{\emph{Научна консултация с предмет ,,Разработване и поддържане на платформа като онлайн уеб-базирана система за езикови ресурси ориентирани към използването на българския книжовен език като официален`` (БЕРОН)}, Финансиран от Министерството на образованието и науката на България}
\cvitem{2021--2022}{\emph{Разработване на синтезатор на българска реч за нуждите на хората със зрителни увреждания}, Финансиран от Съюза на слепите в България}
\cvitem{2018--2022}{\emph{Национална научна програма ,,Електронно здравеопазване в България`` (ННП еЗдраве), Работен пакет 5: ,,Създаване на прототип за компютърно подпомагане на изготвянето на медицинска документация (диктофон за български език)``}, Финансиран от Министерството на образованието и науката на България}

\section{Публикации}
\selectlanguage{english}
\begin{cvenumerate}
	\item Georgi Shopov and Stefan Gerdjikov. \newline \href{https://aclanthology.org/2024.emnlp-main.328/}{Consistent Bidirectional Language Modelling: Expressive Power and Representational Conciseness}. \newline In \emph{Proceedings of the 2024 Conference on Empirical Methods in Natural Language Processing}, 2024.
	\item Diana Geneva, Georgi Shopov and Stoyan Mihov. \newline \href{https://www.worldscientific.com/doi/10.1142/S012905412243002X}{Algorithms for Probabilistic and Stochastic Subsequential Failure Transducers}. \newline In \emph{International Journal of Foundations of Computer Science, 34}(8), 2023.
	\item Diana Geneva, Georgi Shopov, Kostadin Garov, Maria Todorova, Stefan Gerdjikov and Stoyan Mihov. \newline \href{https://www.isca-archive.org/interspeech_2023/geneva23_interspeech.html}{Accentor: An Explicit Lexical Stress Model for TTS Systems}. \newline In \emph{Proceedings of the 24th Annual Conference of the International Speech Communication Association}, 2023.
	\item Georgi Shopov, Stefan Gerdjikov and Stoyan Mihov. \newline \href{https://ieeexplore.ieee.org/document/10096566}{StreamSpeech: Low-Latency Neural Architecture for High-Quality On-Device Speech Synthesis}. \newline In \emph{Proceedings of the 2023 IEEE International Conference on Acoustics, Speech and Signal Processing}, 2023.
	\item Diana Geneva, Georgi Shopov and Stoyan Mihov. \newline \href{https://link.springer.com/chapter/10.1007/978-3-030-79121-6_11}{Algorithms for Probabilistic and Stochastic Subsequential Failure Transducers}. \newline In \emph{Proceedings of the 25th International Conference on Implementation and Application of Automata}, 2021.
	\item Diana Geneva, Georgi Shopov and Stoyan Mihov. \newline \href{https://arxiv.org/abs/2003.09364}{Composition and Weight Pushing of Monotonic Subsequential Failure Transducers Representing Probabilistic Models}. \newline In \emph{arXiv:2003.09364 [cs.FL]}, 2020.
	\item Diana Geneva, Georgi Shopov and Stoyan Mihov. \newline \href{https://link.springer.com/chapter/10.1007/978-3-030-31372-2_16}{Building an ASR Corpus Based on Bulgarian Parliament Speeches}. \newline In \emph{Proceedings of the 7th International Conference on Statistical Language and Speech Processing}, 2019.
	\item Diana Geneva and Georgi Shopov. \newline \href{https://aclanthology.org/R19-2007}{Towards Accurate Text Verbalization for ASR Based on Audio Alignment}. \newline In \emph{Proceedings of the 6th Student Research Workshop Associated with the 12th International Conference on Recent Advances in Natural Language Processing}, 2019.
\end{cvenumerate}

\end{document}
